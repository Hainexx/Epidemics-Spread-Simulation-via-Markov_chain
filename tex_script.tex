\documentclass[10pt]{article}
\usepackage{float}
\usepackage[english]{babel}
\usepackage[utf8]{inputenc}
\usepackage[OT1]{fontenc}
\usepackage{amsfonts, amsmath, amsthm, amssymb}
\usepackage{graphicx}
\usepackage{blkarray}
\usepackage{listings}
\usepackage[margin=1in]{geometry}
\usepackage{xcolor}
\newcounter{countCode}
\lstnewenvironment{code} [1][caption=Ponme caption, label=default]{%
	\renewcommand*{\lstlistingname}{Listado} 
	\setcounter{lstlisting}{\value{countCode}} 
	\lstset{ %
	language=java,
	basicstyle=\ttfamily\footnotesize,       % the size of the fonts that are used for the code
	numbers=left,                   % where to put the line-numbers
	numberstyle=\sc,      % the size of the fonts that are used for the line-numbers
	stepnumber=1,                   % the step between two line-numbers. 
	numbersep=5pt,                 % how far the line-numbers are from the code
	numberstyle=\color{red!50!blue},
    	backgroundcolor=\color{lightgray!20},
	rulecolor=\color{blue},
	keywordstyle=\color{red}\bfseries,
	showspaces=false,               % show spaces adding particular underscores
	showstringspaces=false,         % underline spaces within strings
	showtabs=false,                 % show tabs within strings adding particular underscores
	frame=single,                   % adds a frame around the code
	framexleftmargin=0mm,
	numberblanklines=false,
	xleftmargin=5pt,
	breaklines=true,
	breakatwhitespace=true,
	breakautoindent=true,
	captionpos=t,
	texcl=true,
	tabsize=2,                      % sets default tabsize to 3 spaces
	extendedchars=true,
	inputencoding=utf8, 
	escapechar=\%,
	morekeywords={print, println, size, background, strokeWeight, fill, line, rect, ellipse, triangle, arc, save, PI, HALF_PI, QUARTER_PI, TAU, TWO_PI, width, height,},
	emph=[1]{print,println,}, emphstyle=[1]{\color{blue}}, % Mis palabras clave.
	emph=[2]{width,height,}, emphstyle=[2]{\bf\color{violet}}, % Mis palabras clave.
	emph=[3]{PI, HALF_PI, QUARTER_PI, TAU, TWO_PI}, emphstyle=[3]\color{orange!50!violet}, % Mis palabras clave.
	emph=[4]{line, rect, ellipse, triangle, arc,}, emphstyle=[4]\color{green!70!black}, % Mis palabras clave.
	%emph=[5]{size, background, strokeWeight, fill,}, emphstyle=[5]{\tt \color{red!30!blue}}, % Mis palabras clave.
	%emph={[2]sqrt,baset}, emphstyle={[2]\color{blue}}, % f(sqrt(2)), sqrt a nivel 2 se pondrá azul
	#1}}{\addtocounter{countCode}{1}}

%%%%%%%%%%%%%%%%%%%%%%%%%%%%%%%%%%%%%%%%%%%%%%%%
% Gaspare (Matteo) Staff
\theoremstyle{plain}
\newtheorem{thm}{Theorem} % reset theorem numbering for each chapter

\theoremstyle{definition}
\newtheorem{defn}[thm]{Definition} % definition numbers are dependent on theorem numbers
\usepackage{lmodern,textcomp}
\usepackage{amsmath}

\usepackage{array}
\newcolumntype{L}{>{\displaystyle}l}
\newenvironment{system}%
   {\setlength\arraycolsep{0pt}
    \left\lbrace \begin{array}{L}}%
   {\end{array} \right.}
   
\newcommand{\R}{\mathbb{R}}
\newcommand{\Z}{\mathbb{Z}}
\newcommand{\N}{\mathbb{N}}

\usepackage{caption}
\captionsetup[figure]{
    position=above,
}

    
\usepackage{pgfplots}
\usepackage{tikz}
\usepackage[utf8]{inputenc}

\usepackage{listings}
\usepackage{xcolor}

\definecolor{codegreen}{rgb}{0,0.6,0}
\definecolor{codegray}{rgb}{0.5,0.5,0.5}
\definecolor{codepurple}{rgb}{0.58,0,0.82}
\definecolor{backcolour}{rgb}{0.95,0.95,0.92}

\lstdefinestyle{mystyle}{
    backgroundcolor=\color{backcolour},   
    commentstyle=\color{codegreen},
    keywordstyle=\color{magenta},
    numberstyle=\tiny\color{codegray},
    stringstyle=\color{codepurple},
    basicstyle=\ttfamily\footnotesize,
    breakatwhitespace=false,         
    breaklines=true,                 
    captionpos=b,                    
    keepspaces=true,                 
    numbers=left,                    
    numbersep=5pt,                  
    showspaces=false,                
    showstringspaces=false,
    showtabs=false,                  
    tabsize=2
}

\lstset{style=mystyle}

%%%%%%%%%%%%%%%%%%%%%%%%%%%%%%%%%%%%%%%%%%%%%%%%

\title{\textbf{GTDMO - 3rd Assignment}}
\author{Gaspare Mattarella}
\pgfplotsset{compat = 1.17}
\begin{document}
\maketitle 

\section{Introduction to the Problem}

We can interpret the spread of the epidemics as a stochastic process because the individuals of the population N can assume 3 different states, such as: 
\begin{itemize}

\item  S = Susceptible: not infected and not immune (can get infected) 
\item  I = Infected: the individual has already contracted the virus  
\item  R = Removed: the individual has recovered and become immune, or dead. 
\end{itemize}

At each temporal step the individual can either remain in the same state or move to the following one: 
\begin{itemize}
    \item susceptible individual becomes infected with probability 0.09  
    \item each infected individual is removed with probability 0.27
    \item each removed individual returns back to be susceptible with probability 0.09
\end{itemize}

Supposed that we are considering a population N=10000, our aim is to find the number of individuals in the 3 states after a high number of time steps. 

\section{Problem Interpretation}
This problem can be represented as a Markov Chain as we have 3 different possible states, and each individual can move from a state to another with one time step, moving forward in time.  

As initial distribution we have that each individual belongs to the susceptible state (S). We represent it as an initial state vector: 
\begin{equation}
\textit{s} = \begin{bmatrix}
1 & 0 & 0 \\
\end{bmatrix}
\end{equation}
\\

Given the transition probabilities from one state to the other, our transition matrix is:
\begin{equation}
\begin{blockarray}{cccc}
        & S & I & R \\
      \begin{block}{c(ccc)}
        S & 0.91 & 0.09 & 0 \\
        I & 0 & 0.73 & 0.27 \\
        R & 0.09 & 0 & 0.91\\
      \end{block}
    \end{blockarray}
\end{equation}

Taking into account this Markov Chain, we can now make some considerations about it: 

\begin{itemize}

\item As our set of values of the variable is finite,  \textit{S} = \{S, I, R\}, we can define our Markov Chain as having a finite state space, which means that all the set of values of the random variables is a finite set $\{v_1,v_2,...,v_{n} \}$.


\item As the transition matrix does not change over time, we can define the Markov Chain to be homogeneous.

\end{itemize} 
We can define a Markov Chain as homogeneous if and only if the transition probabilities are independent of the time t, that is, there exist constants $P_{i,j}$ such that:
\begin{equation}
    P_{i,j} =  Pr[X_t =j | X_{t-1} = i]
\end{equation}

holds for all times t. 
\\

To find the number of individuals in the 3 states after a high number of time steps, we should check that the Markov Chain has a stationary distribution. To do so, we should verify if the Markov Chain is irreducible and aperiodic.

A Markov Chain is called irreducible if starting from any state $s_i$ there is a positive probability \((P_{i,j}^{(m)} > 0)\) to reach every other state $s_j$ in a finite $\textit{m}$ number of steps. We can say that in this case every state is \textit{accessible} and belong to only one communication class.\\
By this definition we can state that the Markov Chain we are taking into consideration is irreducible it clearly has this property. \\

For what it concerns the periodicity, we can define the period $d(k)$ of a state $k$ of a homogeneous Markov Chain with transition matrix $P$ as:
\begin{equation}
    d(k) = gcd\{m \geq 1 : P_{i,j}^{(m)} > 0\}
\end{equation}
if $d(k)$ = 1, then we call the state $k$ \textit{aperiodic}.
A Markov chain is aperiodic if and only if all its states are aperiodic.\\
Therefore, for an aperiodic Markov Chain with finite state space, it exists a positive integer N so that  
\begin{equation}
    P_{i,i}^{(m)} > 0
\end{equation}
for every $m\geq N$.

Our Markov chain is clearly aperiodic since the number of steps needed for a state \textit{$s_i$}  to return to state $_i$ is not periodic. 
\\

We can finally state that any aperiodic and irreducible finite Markov chain has precisely one stationary distribution and since our Markov chain appears to have all these properties we can, therefore, assume that it has a stationary distribution, i.e. a probability distribution that remains unchanged as time progresses. In other words, it shows the long term behaviour of the Markov Chain. 

a row vector $\pi$ is said to be a stationary distribution for the Markov Chain, if it satisfies the following conditions:

\begin{itemize}
    \item the $\pi_k$ are nonnegative real numbers such that $\Sigma_{k=0}^{n-1}\pi_k = 1$.
    \item $\pi \textit{P} = \pi$.
\end{itemize}


\newpage
\section{Solution }
A Markov chain with state space $S$ and transition matrix $P$ can be represented by a labeled directed graph G = $(V,E)$ where edges $E$ are given by transitions with nonzero probability. The edge $(u,v)$ is labeled by the probability $P_{u,v}$.

\begin{equation}
    E = \{(u,v)| P_{u,v}>0\}
\end{equation}

\begin{figure}[H]
    \centering
    \includegraphics[scale=0.5]{figures/MCPlot.png}
    \caption{Labeled directed graph}
    \label{fig:my_label1}
\end{figure}
We can establish a long term spread of the virus in the population in the three states S, I and R using the Markov Chain modelling. Therefore we develop the model using R in the following steps: 
\begin{itemize}
    \item Step 1: Create variables\\
    Generate a random number in the states s, with discrete distribution p with the method of the inverse transform where: 

s = vector of states 

p = probability of each state. Vector of the same length as s 
\lstinputlisting[language = R]{figures/step_1.R}
    \item Step 2: Matrix definition \\
    Generate a matrix with the variables created for interaction using Markov Chains modelling
\lstinputlisting[language = R]{figures/step_2.R}
    
    \item Step 3: Markov Chain simulation \\
    Run the simulation using the given parameters
\lstinputlisting[language = R]{figures/step_3.R}
\end{itemize}

After the simulation, we obtain the row vector $\pi$ for the stationary distribution of the Markov Chain 
\begin{equation}
\begin{blockarray}{ccc}
         I & R & S \\
      \begin{block}{(ccc)}
        1322 & 4150 & 4528\\
      \end{block}
    \end{blockarray}
\end{equation}

We can finally confirm its nature by checking that it fits properties of stationary distributions such as
\begin{itemize}
    \item the $\pi_k$ are nonnegative real numbers such that $\Sigma_{k=0}^{n-1}\pi_k = 1$.
    \item $\pi \textit{P} = \pi$.
\end{itemize}

In conclusion, we can state that in the long run, the epidemic spread will asses itself on a stationary distribution having the 13,9\% of the population Infected, 43,5\% of the population Removed and 42,4\% of the population Susceptible again. With no further information update in the model, this distribution will remain stable over time.
\newpage
\subsection{Appendix}
For the sake of Reproducibility, here I attach the full commented R code used to obtain all Figures, Tables, equations and results.
\lstinputlisting[language = R]{Assign3.R}
\end{document}
